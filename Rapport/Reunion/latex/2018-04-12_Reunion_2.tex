\documentclass[11pt]{meetingmins}
\usepackage[french]{babel}
\usepackage[T1]{fontenc}
\usepackage[utf8]{inputenc}
\setcommittee{Compte rendu de réunion 2}

\setdate{April 12, 2018, 12h00 - 50mins}
\setmembers{
	Nathan BARLOY,
	\chair {Johan TOMBRE},
	Valentin CRÔNE
}
\setpresent{
	Nathan BARLOY,
	\chair {Johan TOMBRE},
	Valentin CRÔNE
}
\absent{aucun}

\begin{document}
\maketitle

\section{Ordre du jour}
Les éléments discutés lors de cette réunion ont été :
\begin{hiddenitems}
	\item L'avancement du projet, ce qui comprend :
	\begin{itemize}
		\item Détails sur le code produit, fonctionnement des librairies.
		\item Discussion sur la structure interne des différents objets.
		\item Choix du type d'interface graphique à utiliser.
	\end{itemize}
	\item Rappel sur l'importance d'un travail régulier.
	\item Discussion sur comment être plus efficace sur ce projet.
\end{hiddenitems}

\section {Compte rendu}
	La base de données a été transformé en JSON (Valentin) selon un schéma décrit dans la structure interne.
	Il a également été décidé que ce format de données sera celui utilisé pour toutes les informations à stocker. Valentin a donc expliqué au reste du groupe le fonctionnement de la librairie libcjson que nous utiliserons à l'avenir.\par
	Nous avons ensuite échangé sur la structure interne de la base de donnée JSON du serveur.
	Nathan a fait remarquer qu'il est intéressant de ne calculer qu'une seule fois les similarités entre films au lancement du serveur puis de les stocker.\par
	Le client nous a donné son accord pour l'utilisation de libcommon, libcjson, et libjson, un modèle client/serveur multiutilisateur, ainsi qu'un client graphique utilisant WebkitWebView de WebkitGTK+.\par
	Nous avons conclu par un recadrage de l'équipe suite au déséquilibre de la proportion du travail effectué par chaque membre du groupe. Il a été décidé qu'il fallait effectuer un travail plus régulier. De plus, le groupe se réunira à présent 2 fois par semaine lors des réunions plus courtes pour faire le point ainsi qu'établir de nouveaux objectifs (réalistes) à réaliser pour la prochaine fois.

\section{A faire pour la prochaine réunion}
\begin{items}
	\item Rédiger le compte rendu propre de la réunion (Johan TOMBRE)
	\item Amorcer le code pour la mesure de similarités entre films (Nathan BARLOY)
	\item Rédaction de l'état de l'art (Johan TOMBRE)
	\item Continuer le code serveur / client (Valentin CRÔNE)
\end{items}
\vspace{1cm}
\nextmeeting {April 16, 2018, 12h00}

\end{document}
