\documentclass[11pt]{meetingmins}
\usepackage[french]{babel}
\usepackage[T1]{fontenc}
\usepackage[utf8]{inputenc}
\setcommittee{Compte rendu de réunion 6}

\setdate{May 14, 2018, 17h - 25mins}
\setmembers{
	Nathan BARLOY,
	\chair {Johan TOMBRE},
	Valentin CRÔNE
}
\setpresent{
	Nathan BARLOY,
	\chair {Johan TOMBRE},
	Valentin CRÔNE
}
\absent{aucun}

\begin{document}
\maketitle

\section{Ordre du jour}
\begin{hiddenitems}
	\item Montrer les avancés sur l'interface graphique
	\item Explication et démonstration des codes de requêtes
	\item Répartition des tâches sur la fin du projet
\end{hiddenitems}
\section{Compte rendu}
Valentin a codé la base de l'interface graphique qui consiste à générer du code html à partir de fonctions en C pour être utilisé avec la libraire GTKWebKitWebView.\par
Johan a enrichit la base de données avec les affiches des films à l'aide d'un script python.\par
Nathan a réalisé le code permettant un calcul de la distance entre genres à partir d'une matrice.
\vspace{0.5cm}
\section{A faire}
Les attributions des tâches sont :
\begin{items}
	\item Générer une base de données utilisateur afin de tester l'algorithme de distance entre utilisateurs (Johan TOMBRE)
	\item Finir l'algorithme collaboratif = estimations de notes et tri des résultats (Johan TOMBRE)
	\item Réalisation des outils de gestion de projet (Nathan BARLOY)
	\item Réalisation de l'interface graphique (Valentin CRÔNE et Johan TOMBRE)
	\item Réalisation interface console (Valentin CRÔNE)
	\item Gestion des exports (Valentin CRÔNE)
	\item Rédaction du rapport (Tous)
\end{items}

\end{document}
