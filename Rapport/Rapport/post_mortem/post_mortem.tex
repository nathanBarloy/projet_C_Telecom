\section{Post-mortem}
L'objectif de ce projet était de réaliser un système de recommandations de films. Le code fournit doit être produit en C.\par
Après un peu plus de 2 mois, notre code comprend :
\begin{itemize}
	\item Une interaction client / serveur
	\item Une base de données au format JSON
	\item Un système de recommandation basé sur le contenu
	\item Un système de recommandation collaboratif
	\item Une interface console
	\item Une interface graphique
\end{itemize}
\vspace{0.5cm}
L'ensemble fonctionne correctement et présente l'avantage d'être modulaire : il est donc possible de faire évoluer le programme afin d'y inclure davantage de fonctionnalités à l'avenir.\par
L'ensemble du groupe est satisfait du résultat obtenu. Les éléments ayant bien fonctionnés au cours de ce projet sont le worflow utilisé (très effectif) ainsi que le code produit de façon très modulaire.\par
Toutefois, nous pouvons également soulever quelques aspects ayant moins bien fonctionnés, comme par exemple le lancement du projet. En effet, nous ne nous connaissions pas avant le début du projet et avons eu du mal à travailler ensemble et à nous répartir les tâches correctement. Cela s'est toutefois grandement amélioré au cours du projet, avec notamment la mise en place de réunions plus fréquentes.\par
Pour conclure, ce projet a été très bénéfique pour l'ensemble du groupe, tant sur le plan didactique que relationnel.
