% \documentclass{report}
% \begin{document}
\section{Structures de données}
Les structures de données du projet peuvent être séparées en 2 types:
\begin{itemize}
	\item Statiques
	\item Dynamiques
\end{itemize}
Les structures statiques sont celles qui sont directement inscrites dans le code du programme C, écrites en dur dans le code, ce sont des structures de bas niveau. Ce sont les structures que l'on va définir avec le mot-clé struct en C, et qui sont destinées à contenir des données connues par avance, un tableau, une autre structure, des variables, etc...\par
Les structures dynamiques sont celles qui utilisent une autre structure plus flexible et qui permet l'agencement de données pour créer d'autres structures de données. Ce sont les structures basées sur le JSON, qui vont utiliser libcjson, et qui permettent d'ajouter et supprimer dynamiquement des éléments via des fonctions. Pour ces stuctures nous avons défini un fichier de référence en Markdown sur le dépôt git, afin que le programme utilise un format uniforme de données, mais une structure JSON permet en réalité de représenter n'importe quelle structure de données.\par
% \end{document}
