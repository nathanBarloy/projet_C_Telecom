% \documentclass{report}
% \begin{document}
\subsection{Structures de données statiques}
\subsubsection{Structures provenant de libcommon}
Les structures suivantes proviennent de libcommon:
\begin{itemize}
	\item String\_t : Structure représentant une chaine de caractère, avec de nombreuses fonctions associées
	\item AutoString\_t : Structure identique à String\_t, pointeur automatiquement géré en mémoire.
	\item Map\_t : HashMap permettant l'association entre un String\_t et un void*
	\item Vector\_t : Tableau dynamique permettant le stockage de void*
\end{itemize}
La documentation détaillée de libcommon se trouve à cette adresse:\newline
https://tanaka1238.ovh/libcommon-doc
\par
\subsubsection{Structures provenant de libcjson}
Les structures suivantes proviennent de libcjson:
\begin{itemize}
	\item JSONObject\_t : Représente un objet JSON (associatif)
	\item JSONArray\_t : Représente un tableau JSON (dynamique)
	\item JSONInt\_t : Représente un nombre entier en JSON
	\item JSONDouble\_t : Représente un nombre flottant en JSON
	\item JSONString\_t : Représente une chaine de caractère en JSON
	\item JSONNull\_t : Représente une valeur nulle en JSON
	\item JSONBoolean\_t : Représente une valeur booléene en JSON
\end{itemize}
Ces structures sont utilisées pour créer les structures dynamiques expliquées plus bas.
Elles permettent aussi de parser un JSON, le manipuler, et exporter les données.\par

La documentation détaillée se trouve à cette adresse:\newline
https://tanaka1238.ovh/libcjson-doc/
\subsubsection{Base de données}
\begin{lstlisting}
struct BDD
{
	JSONObject_t json;
	pthread_mutex_t mutex;
	Vector_t clients;
	Map_t requests;
};
typedef struct BDD* BDD;
\end{lstlisting}


% \end{document}
