\section{Amélioration de la base de données}
Dans le but d'améliorer l'expérience de l'utilisateur sur l'interface graphique, nous avons décider de compléter la base de données mise à disposition afin d'y inclure les liens vers les bandes annonces des films ainsi que leurs affiches respectives.
De plus, le fonctionnement de l'algorithme de recommandation collaboratif requière une base de données d'utilisateurs ayant notés plusieurs films.
\par Nous avons décidé de réaliser ces opérations à partir de scripts python qui sont plus simple et rapide à mettre en oeuvre, grâce notamment à ses nombreuses librairies disponibles en ligne.

\subsection{Liens des bandes annonces}
Comme notre interface graphique sera gérée grâce à la librairie WebKitWebView, il sera possible d'intégrer à notre page un lecteur YouTube avec la bande annonce.
Il suffit donc de récupérer les liens URL des vidéos pour que l'ensemble fonctionne. Ces résultats seront alors stockés dans un fichier texte avant d'inclure les liens au fichier JSON.
\vspace{0.5cm}
Ce script se décompose en deux étapes.\par
Tout d'abord, il faut réussir à générer une requête à partir du nom du film. Pour cela, nous nous appuyons sur la simplicité de l'URL lors d'une recherche. Cette URL possède une partie constante qui est \newline https://www.youtube.com/results?search\_query=
suivi du contenu de la recherche avec des "+" remplaçant les espaces.
Notre requête est donc composé du nom du film, récupéré à partir du fichier JSON suivi des mots "bande" et "annonce". Il faut considérer certains cas où le titre du film contient un caractère interdit dans une URL (comme le \&). Ce caractère doit dans ce cas être encodé à l'aide du signe \% suivi d'un nombre (dans notre cas le 26)
Une fois l'URL de la requête généré, elle est exécuté pour récupérer le code HTML de la page des résultats.\par
La seconde étape consiste à récupérer le lien de la première vidéo proposée (il est supposé que l'algorithme de YouTube effectue bien sa mission). Il est intéressant de remarquer qu'un lien YouTube ne se distingue que par ses 11 derniers caractères. Nous utilisons donc une expression régulière qui recherche tous les liens des vidéos proposées puis on stocke ces séries de 11 caractères dans une liste.
Pour terminer, on crée le lien de la vidéo en concaténant le début du lien https://www.youtube.com/watch?v= et la première série de caractère trouvé. Il ne reste plus qu'à stocker ce lien à côté de l'ID du film dans un fichier texte.

\subsection{Récupération des affiches des films}
Avoir une affiche de chaque film de la base de données permet une fois de plus d'améliorer grandement l'interface graphique.
Après plusieurs recherches, nous avons décidé de récupérer les images du site \href{https://www.themoviedb.org}[TheMovieDB].\par
Pour ce faire nous avons, de la même manière que pour les liens Youtube utilisé les url pour effectuer nos recherches puis nous avons scanné le contenu de la page html résultante pour détecter le résultat correspond à notre recherche (correspondance des dates notamment). Une fois trouvé, nous naviguons vers la page du résultat et scannons à nouveau la page à la recherche de l'image souhaité avant de la télécharger.

\subsection{Création d'une base de données utilisateurs}
Indispensable au fonctionnement de la méthode de recommandation collaborative, une base de données d'uilisateurs ayant attribués des notes aux films était nécessaire. Plusieurs pistes ont été explorées :
\begin{itemize}
	\item création fictive d'utilisateurs et de notes (aléatoires)
	\item création d'un formulaire à partager avec des amis pour constituer notre propre base de données
	\item trouver ces informations en ligne
\end{itemize}
Nous avons au final retenu cette dernière solution après avoir trouvé une base de données de \href{https://grouplens.org/datasets/movielens/}[MovieLens] qui regroupe près de 100000 notes d'utilisateurs sur 10000 films. Le seul inconvénient est que cette base de données ne prenne pas en compte les séries. Par conséquent, les résultats exposés ne seront des plus justes mais permettra d'avoir un aperçu de cette algorithme.\par
La base de données de \href{https://grouplens.org/datasets/movielens/}[MovieLens] est au format csv, il était donc très simple de récupérer les données souhaitées. Le seul problème était les titres des films qui étaient en anglais et ne correspondaient pas toujours à nos titres. Il a donc fallu prendre quelques minutes pour manuellement faire correspondre leurs titres avec nos id de film. Ensuite il n'est plus que question d'extraire les bons éléments du fichier csv et le tour est joué !
