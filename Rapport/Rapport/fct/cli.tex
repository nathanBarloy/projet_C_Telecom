% \documentclass{report}
% \begin{document}
\subsection{Fonctionnement du client CLI}
Le client en console nécessite d'avoir en paramètre l'ip ou le nom de domaine du serveur auquel se connecter. Dès son lancement, il vérifie que le serveur est joignable, puis affiche le menu principal.\par
Les menus de l'interface s'inspirent d'un fonctionnement \og cisco \fg, il suffit d'entrer le début d'un mot affiché pour que la commande complète soit reconnue, si plusieurs mots correspondent, l'interface affiche les choix possible et demande à l'utilisateur de préciser sa demande.\par
L'interface console permet de faire autant de choses que l'interface graphique, elle permet de consulter les films, voir les tendances, voir des recommandations personnalisées, voir les films qui ont été notés, attribuer une note à des films, etc...\par
Les fonctions du menu principal sont dans un tableau et sont activées par un pointeur de fonction, le but de ce procédé est de rendre le code plus modulaire.\par
% \end{document}
